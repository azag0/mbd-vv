\section{Introduction}

The tree of 

\section{Theory}

The basic proposition of our new model, dubbed MBD@VV, is to parameterize the many-body dispersion (MBD) Hamiltonian with the Vydrov–Van Voorhis (VV) polarizability functional.
The MBD Hamiltonian \citep{TkatchenkoPRL12,TkatchenkoJCP13,Blood-ForsytheCS16} describes a system of harmonic oscillators $\boldsymbol\xi_i$ characterized by their static polarizabilities $\alpha_{0,i}$ and frequencies $\omega_i$ and interacting via a long-range dipole potential $\mathbf T^\text{lr}$,
\begin{multline}
  H^\text{MBD}(\{\alpha_{0,i},\omega_i\})=\sum_i-\frac12\nabla_{\xi_i}^2+\sum_i\frac12\omega_i^2\xi_i^2 \\
  +\frac12\sum_{ij}\omega_i\omega_j\sqrt{\alpha_{0,i}\alpha_{0,j}}\boldsymbol{\xi}_i\cdot\mathbf T^\text{lr}_{ij}\boldsymbol{\xi}_j
\end{multline}
We obtain the oscillator parameters by coarse-graining \citep{SatoJCP09,SatoJCP10} the VV polarizability functional \citep{VydrovPRL09,VydrovJCP10a,VydrovPRA10},
\begin{equation}
   \alpha^\text{VV}[n](\mathrm iu)=\frac{n}{\frac{4\pi}3n+C\frac{|\boldsymbol\nabla n|^4}{n^4}+u^2}
   \label{eq:vv-functional}
\end{equation}
to Hirshfeld fragments $w_i^\text{H}$ \citep{HirshfeldTCA77},
\begin{equation}
  \alpha_i^\text{VV}(\mathrm iu)=\int\mathrm d\mathbf r\,w_i^\text{H}(\mathbf r)\alpha^\text{VV}[n](\mathbf r,\mathrm iu)
\end{equation}
The effective oscillator frequencies are calculated such as to reproduce the same $C_6$ coefficients for the oscillators as if calculated directly from $\alpha_i(\mathrm iu)$ via the Casimir--Polder formula,
\begin{equation}
  \omega_i=\frac43\frac{C_{6,i}}{\alpha_{0,i}^2},\quad
  C_{6,i}=\frac3\pi\int_0^\infty\mathrm du\,\alpha_i(\mathrm iu)^2
\end{equation}

This plain combination of the MBD approach and VV polarizability functional already improves description of ionic systems compared to previous MBD parameterizations.
But when combined with DFT, it suffers from double-counting electron correlation in metallic systems (slowly-varying electron density regions), and lacks in accuracy for covalent systems with respect to state-of-the-art vdW methods.
To solve these two issues, we borrow two techniques from the VV10 nonlocal functional and the TS method.
First, we subtract the portion of the polarizability that comes from metallic electron density regions.
Second, we normalize the atomic VV polarizabilities and $C_6$ coefficients to reproduce the respective exact quantities for free atoms.

\begin{figure}[t!]
\centering
\includegraphics{../media/solids-hists-cutoff.pdf}
\caption{\textbf{Distributions of the reduced gradient $s$ and iso-orbital indicator $\alpha$ according to contribution to the total polarizability.}
The five blackish plots show the distributions $\alpha^\text{VV}(s,\alpha)$ aggregated over multiple crystals within each group.
The magnitude of the distribution is mapped to the color intensity on a log scale.
The bottom-right plot shows the function $g_\text{nm}$ used to distinguish the nonmetallic density regions.
The crystals in each group as well as details of the calculations can be found in Supplementary Information.
}\label{fig:solids-hists-cutoff}
\end{figure}

Most exchange--correlation functionals are exact for the uniform electron gas by construction and as a result, describe accurately the electron correlation \emph{within} slowly-varying density regions, which can be found in metals.
Furthermore, the interactions \emph{between} such regions and other regions of the electron density are effectively screened by the conducting electrons.
At the same time, these metallic-density regions contribute dominantly to the total polarizability (see eq.~\ref{eq:vv-functional}) and therefore, if used directly in any vdW model, would result in overpolarization and overbinding of metallic systems.
To avoid this double-counting of the electron correlation, we smoothly cut off the VV polarizability functional in metallic-density regions.
These regions can be distinguished using the combination of two local electron-density descriptors: the reduced gradient $s$ and the iso-orbital indicator $\alpha$ \citep{BeckeJCP90,KummelMP03,SunPRL13},
\begin{equation}
  s[n]=\frac{|\boldsymbol\nabla n|}{2(3\pi^2)^\frac13n^\frac43},\quad
  \alpha[n]=\frac{\tau-\tau_\text{W}}{\tau^\text{unif}}
\end{equation}
In particular, the metallic density is characterized by $s\rightarrow0$ and $\alpha\sim1$.
Based on the numerical analysis of these two descriptors in a wide range of materials, we devised a function of these two descriptors that distinguishes the nonmetallic (nm) density regions (Fig.~\ref{fig:solids-hists-cutoff}),
\begin{equation}
  g_\text{nm}(s,\alpha)=\ldots,\quad
  \alpha^\text{nmVV}[n]=g_\text{nm}(s,\alpha)\alpha^\text{VV}[n]
\end{equation}
The exact shape of the function is to some degree arbitrary, but mostly can be characterized as being zero in the largest possible neighborhood around $(s,\alpha)=(0,1)$ without making $\alpha^\text{nmVV}[n]$ significantly smaller than $\alpha^\text{VV}[n]$ for the group of semiconductors.
We discuss the sensitivity of our model to the shape of $g_\text{nm}$ in greater detail below.

The VV polarizability functional is only approximate, which is manifest already for free-atom polarizabilities and $C_6$ coefficients, where accurate reference values are known.
To mitigate this error, we normalize the VV quantities with the ratio of the free-atom polarizabilities and $C_6$ coefficients as calculated by the VV functional and as given by reference calculations,
\begin{equation}
  \alpha_{0,i}^\text{rVV}=\alpha_{0,i}^\text{nmVV}\frac{\alpha_{0,i}^\text{ref,free}}{\alpha_{0,i}^\text{VV,free}},\quad
  C_{6,i}^\text{rVV}=C_{6,i}^\text{nmVV}\frac{C_{6,i}^\text{ref,free}}{C_{6,i}^\text{VV,free}}
\end{equation}
This correction assumes that any error in $\alpha^\text{VV}[n]$ (or $\alpha^\text{nmVV}[n]$) is at least partially transferable from free atoms to atoms in molecules and materials.
Alternatively, this step can be seen as a modification of the TS method \citep{TkatchenkoPRL09}, in which the scaling of reference free-atom quantities by Hirshfeld volumes is replaced with that by VV-derived polarizabilities and $C_6$ coefficients.

Finally, we use the same definition of $\mathbf T^\text{lr}$ as in the MBD@rsSCS variant \citep{AmbrosettiJCP14},
\begin{equation}
  \mathbf T_{ij}^\text{lr}=\frac1{1+\exp\left(-6\Big(\frac{|\mathbf r|}{\beta(R_i^\text{vdw}+R_j^\text{vdw})}-1\Big)\!\!\right)}\boldsymbol\nabla\otimes\boldsymbol\nabla\frac1{|\mathbf r|}\Bigg|_{\mathbf r=\mathbf R_j-\mathbf R_i}
\end{equation}
with the vdW radii $R_i^\text{vdW}$ obtained by scaling free-atom radii with ratio of Hirshfeld volumes in the molecule or material and in free atoms,
\begin{equation}
  R_i^\text{vdW}=R_i^\text{vdW,free}\left(\frac{V_i^\text{H}}{V_i^\text{H,free}}\right)^\frac13,\quad
  R_i^\text{vdW,free}=\tfrac52(\alpha_{0,i}^\text{free})^\frac17
\end{equation}
The only novel minor modification is that the free-atom radii are calculated from free-atom static polarizabilities \citep{Dima} rather than taken as external input.

\section{Numerical results}

\section{Discussion}

Hirshfeld partitioning/Hirshfeld volumes, sensitivity to choice of partitioning.

How double-counting is treated

Discuss short/long-range nature of fluctuations in metals, screening.

Discuss $g_\text{nm}$.

\section{Conclusions}
